This thesis is the final chapter of my Computer Science and Engineering Master's Degree.
Deciding the thesis subject was not an easy choice.
I was really into the development of some software in \glspl{distributed cloud} related topics.
There were a few different options proposed to me, worth noting:
one was akin to \glspl{compiler};
another one was related to \gls{systems security} with \gls{SELinux}.

In order to bring together the interest for \glspl{distributed system} and the options on the table, I went for an exploratory work in \Gls{container} Isolation with \gls{SELinux} in \gls{Kubernetes}, \gls{OpenShift} Clusters on the supervision of \mysupervisor.
While the topic was engaging, dense of new knowledge and intensively under \gls{continuous development}, these properties, at least for me, made its \gls{learning curve} pretty steep.
I had to pivot.

The new chosen topic was on the usage of relatively new database systems to solve a real life problem.
The problem on which to work was the individuation of the collaboration communities in academia.
Using graph databases and a community detection algorithm made perfect sense - this way, this thesis was framed in its present shape.

\section{About this thesis} \label{section:Preface/Aboutthisthesis}
This document describes the work on the thesis done during the ending of the \gls{Masters Degree}.%Master's Degree
The work started before mid 2021, was initially exploratory and was carried out under the supervision of \mysupervisor.

\subsection{What this thesis covers} \label{subsection:Preface/Aboutthisthesis/Whatthisthesiscovers}
The topic this thesis treats is the application of a \gls{Label Propagation} Community Detection algorithm to detect the collaboration communities in the academia, between researchers, their publications and so on...

The data on academic publications was obtained from a complete dataset of \citelink[]{Dagstuhl2021} of \citeauthor{Dagstuhl2021} \sfcite{Dagstuhl2021}.
The version of the dataset is of the 1\textsuperscript{st} of July 2021.

In order to host the data and do the necessary on graph calculations, a Graph Database Management System was used, specifically ArangoDB.

To query the data and visualize the results, an \acrshort{API} and a \gls{frontend} interface were built.

For more details, head to \hyperref[chapter:Introduction]{\S\ \ref{chapter:Introduction} - \nameref{chapter:Introduction}}.

\subsection{Get and run the code} \label{subsection:Preface/Aboutthisthesis/Getandrunthecode}
\paragraph{Pull the source code}\mbox{}\\\indent
The instructions to get the source code used for the \gls{data manipulation}, the \gls{import}ing in the database, the further collection updates, the \gls{GraphQL} \acrshort{API} and the frontend - are in \hyperref[appendix:SourceCode]{appendix \S\ \ref{appendix:SourceCode} - \nameref{appendix:SourceCode}}, specifically \hyperref[section:SourceCode/Projectrepositories]{\S\ \ref{section:SourceCode/Projectrepositories} - \nameref{section:SourceCode/Projectrepositories}} on \hyperref[section:SourceCode/Projectrepositories]{page \pageref{section:SourceCode/Projectrepositories}}.
\par
\paragraph{Run, build and deploy}\mbox{}\\\indent
The instructions on how to get everything running are in \hyperref[section:SourceCode/Instructionshowtorunbuildanddeploy]{\S\ \ref{section:SourceCode/Instructionshowtorunbuildanddeploy} - \nameref{section:SourceCode/Instructionshowtorunbuildanddeploy}} on \hyperref[section:SourceCode/Instructionshowtorunbuildanddeploy]{page \pageref{section:SourceCode/Instructionshowtorunbuildanddeploy}}.
\par

\subsection{Notation conventions used} \label{subsection:Preface/Aboutthisthesis/Notationconventionsused}
\noindent\textit{Italic}

\noindent Indicates new terms or terms that are non part of standard vocabulary.
\bigskip

\noindent\texttt{Constant width, monospace} (in black color)

\noindent Used for code listings, as well as within paragraphs to refer to program elements such as variable or function names, databases, data types and instructions.
\bigskip

\noindent\href{https://en.wikipedia.org/wiki/URL}{\texttt{Constant width, monospace}} or \href{https://en.wikipedia.org/wiki/URL}{normal fontstyle} (in blue color)

Used for links, urls, websites
\bigskip

\noindent\sfcite{LogoUnibg2021}

\noindent Used for citations.
Clicking on the number sends to \hyperref[Backmatter:Bibliography]{Bibliography} on \hyperref[Backmatter:Bibliography]{page \pageref{Backmatter:Bibliography}}.
In Bibliography are also indicated all the pages where each reference has been cited.
The bibliographic information of a reference is shown in footnotes of a page (see the bottom of the page) only the first time it is cited. Numbering and sorting of references in the bibliography is in ascending order of their citation.
\bigskip

\noindent This is\footnote{this is a footnote}

\noindent Used for simple footnotes\footnote{not a bibliographic citation}.
Differently from bibliographic citations, footnotes use superscripted smallcase letters\footnote{uses letters, not numbers} in green color inside round brackets.
\bigskip

%\noindent\Gls{graph}

%\noindent Komejirushi (reference mark), used to indicate terms part of the glossary or acronyms.
%It is hyperlinked, following the link sends to the glossary or acronyms index where is possible to see the other pages the term appears in.
%\bigskip

\noindent\S\ 2.1

\noindent Silcrow (section sign), used to refer to a section or subsection of the document.
\bigskip

%\noindent\P\ 2.2.2.2.1

%\noindent Pilcrow (paragraph sign), used to refer to a paragraph in the document.
%\bigskip

\noindent\hyperref[Listoffigures]{\footnotesize\faListUl}\ \ Figure 1.2: A random title

\noindent FontAwesome list icon, used to show a figure (or table, code listing, definition, theorem, proof or remark) is included in the list or figures (or list of tables, of code listings and so on ...).
It is hyperlinked, clicking on it sends to the list of that element type.
\bigskip\smallskip

\noindent\begin{tabularx}{1\textwidth}{ l >{\raggedright\arraybackslash}X }%
	\rowcolor{lightestgray}%
	\rule[-6pt]{-11.5pt}{15pt} & % to fix heigth of the row
	\hyperref[Index]{Index}
	\adforn{43}
	\hyperlink{\chaptername::\thechapter}{\thechapter\ \nameref{\chaptername::\thechapter}}
	\adforn{43}
	\hyperlink{section:Preface/Aboutthisthesis}{\ref{section:Preface/Aboutthisthesis} \nameref{section:Preface/Aboutthisthesis}}
	\adforn{43}
	\hyperlink{subsection:Preface/Aboutthisthesis/Notationconventionsused}{\ref{subsection:Preface/Aboutthisthesis/Notationconventionsused} \nameref{subsection:Preface/Aboutthisthesis/Notationconventionsused}}\\%
\end{tabularx}

\noindent Used for breadcrumbs in the header of every page except for the frontmatter and chapters, appendices and bibliography's first page.
Other then general orientational information on the current topic and position in the chapter, breadcrumbs offer also interactive browsing of the whole document.

\noindent Clicking on \hyperref[Index]{Index} sends to the general contents index.

\noindent Clicking on \hyperlink{\chaptername::\thechapter}{\thechapter\ \nameref{\chaptername::\thechapter}} sends to the current chapters' first page, where its subcontents can be consulted.

\noindent Clicking on \hyperlink{section:Preface/Aboutthisthesis}{\ref{section:Preface/Aboutthisthesis} \nameref{section:Preface/Aboutthisthesis}} sends to the current sections' beginning.

\noindent Clicking on \hyperlink{subsection:Preface/Aboutthisthesis/Notationconventionsused}{\ref{subsection:Preface/Aboutthisthesis/Notationconventionsused} \nameref{subsection:Preface/Aboutthisthesis/Notationconventionsused}} sends to the current subsections' beginning.

\subsection{About the cover illustration} \label{subsection:Preface/Aboutthisthesis/Aboutthecoverillustration}
On the \hyperref[fig:cover]{cover} and the \hyperref[fig:backcover]{backcover} of the thesis is a picture from \citefirstlastauthor{Clode2018} taken in \citeyear{Clode2018} with caption "\citetitle{Clode2018}". \sfcite{Clode2018}

Symbolically knots can be intended as nodes of a graph, ropes as the edges and the different coloring of these intended as the different communities of that graph.

\subsection{About illustrations on the first page of each chapter or appendix}
\label{subsection:Preface/Aboutthisthesis/Aboutillustrationsonthefirstpageofeachchapterorappendix}
Icons placed on the first page of each chapter or appendix were obtained from \citeurl{UXWing2020}. \sfcite{UXWing2020}

\subsection{Graph of order of reading the chapters} \label{subsection:Preface/Aboutthisthesis/Graphoforderofreadingthechapters}
	Let Chapter X be the begin and Chapter Y the end node of an edge displayed below.
	
\noindent If the edge is a \textit{continuous arrow}, in order to read (and well understand) Chapter Y, is \textit{required} to have read Chapter X beforehand.

\noindent If the edge is a \textit{dashed arrow}, in order to read (and well understand) Chapter Y, is \textit{recommended} to have read Chapter X beforehand.

\begin{figure}[H]%
	\centering%
	\adjustbox{
		max width=1\linewidth-8pt,
		margin*=2pt,
		cframe=lightestgray 2pt 0pt 0pt,
		bgcolor=white
	}{
		\begin{tikzpicture}[%
			>=stealth,%
			every node/.style={
				text width=1.75cm,
				align=center,
				color=white,
				draw
			}
		]
			
			% create the nodes,
			\node (c0) [fill=lightstgray, shape=rectangle, rounded corners] {
				\hyperref[Preface]{%
					\ref{Preface}\\%
					\nameref{Preface}%
				}%
			};
			
			\node (c1) [fill=lightergray, shape=circle] [
				below =of c0
			] {
				\hyperref[chapter:Introduction]{%
					Chapter %
					\ref{chapter:Introduction}\\%
					\nameref{chapter:Introduction}%
				}%
			};
			
			\node (c2) [fill=lightstgray, shape=circle] [
				right =of $(c1)!0.67!(c1.east)$
			] {
				\hyperref[chapter:LiteratureReview]{%
					Chapter %
					\ref{chapter:LiteratureReview}\\%
					\nameref{chapter:LiteratureReview}%
				}%
			};
			
			\node (c3) [fill=lightstgray, shape=circle] [
				right =of $(c2)!0.67!(c2.east)$
			] {
				\hyperref[chapter:CommunityDetection]{%
					Chapter %
					\ref{chapter:CommunityDetection}\\%
					\nameref{chapter:CommunityDetection}%
				}%
			};
			
			\node (c4) [fill=lightstgray, shape=circle] [
				right =of $(c3)!0.67!(c3.east)$
			] {
				\hyperref[chapter:ImplementingtheWebApp]{%
					Chapter %
					\ref{chapter:ImplementingtheWebApp}\\%
					\nameref{chapter:ImplementingtheWebApp}%
				}%
			};
			
			\node (c5) [fill=lightstgray, shape=circle] [
				right =of $(c4)!0.67!(c4.east)$
			] {
				\hyperref[chapter:Displayoftheresults]{Chapter %
				\ref{chapter:Displayoftheresults}\\%
				\nameref{chapter:Displayoftheresults}%
			}%
		};
			
			\node (c6) [fill=lightergray, shape=circle] [
				right =of $(c5)!0.67!(c5.east)$
			] {
				\hyperref[chapter:Conclusions]{Chapter \ref{chapter:Conclusions}\\
				\nameref{chapter:Conclusions}}
			};
			
			\node (c8) [fill=lightstgray, shape=rectangle, rounded corners] [
				below =of c6
			] {
				\hyperref[appendix:APIDocs]{Appendix \ref{appendix:APIDocs}\\
				\nameref{appendix:APIDocs}}
			};
			
			\node (c7) [fill=lightstgray, shape=rectangle, rounded corners] [
				left =of $(c8.west)!0.15!(c8)$
			] {
				\hyperref[appendix:SourceCode]{%
					Appendix %
					\ref{appendix:SourceCode}\\%
					\nameref{appendix:SourceCode}%
				}%
			};
			
			\node (c9) [fill=lightstgray, shape=rectangle, rounded corners] [
				below =of $(c6.south)!0.67!(c8.south)$
			] {
				\hyperref[Backmatter:Bibliography]{Bibliography}
			};
			
			% connect the nodes
			\draw[->, dashed, color=darkgray] (c0)			to					(c1);%
			\draw[->,		 color=darkgray] (c1)			to					(c2);%
			\draw[->,		 color=darkgray] (c1)			to[out=45, in=135]	(c3);%
			\draw[->, dashed, color=darkgray] (c2)			to					(c3);%
			\draw[->,		 color=darkgray] (c3)			to					(c4);%
			\draw[->,		 color=darkgray] (c3)			to[out=45, in=135]	(c5);%
			\draw[->, dashed, color=darkgray] (c4)			to					(c5);%
			\draw[->,		 color=darkgray] (c5)			to					(c6);%
			\draw[->, dashed, color=darkgray] (c7.north west) to[out=135, in=307.5] (c4);%
			\draw[->, dashed, color=darkgray] (c8.north west) to[out=145, in=320]   (c4);%
			\draw[->, dashed, color=darkgray] (c9.west)	   to[out=180, in=315]   (c2.south east);%
		\end{tikzpicture}%
	}
	\caption[Graph of order of reading the chapters]{Graph of order of reading the chapters (hyperlinked titles)}
	\label{fig:Graphoforderofreadingthechapters}%
\end{figure}

\subsection{Technologies and tools used} \label{subsection:Preface/Aboutthisthesis/Technologiesandtoolsused}
During the work for this thesis, the following technologies and tools were used:
\setlist{nolistsep} \begin{itemize}[noitemsep]
	\item \gls{CentOS} and \gls{Ubuntu};
	\item ArangoDB and temporarily Neo4j;
	\item \gls{JavaScript}, \gls{Python}, \acrshort{AQL}, \gls{NodeJS}, \gls{React}, improperly \gls{TypeScript}, \Gls{bootstrap}, \gls{Express}, \gls{GraphQL}, \gls{Apollo}, \gls{Cytoscape.JS} and many more;
	\item \gls{Visual Studio Code}, \gls{WebStorm}, \gls{PyCharm}, \gls{Postman}, \texttt{\gls{vim}} and \texttt{\gls{gedit}};
	\item \texttt{\gls{git}}, \texttt{\gls{konsole}} and \gls{GitHub};
	\item \gls{Evince} Document Viewer and \gls{Calibre};
	\item \gls{Firefox} and \gls{Chromium} browsers;
\end{itemize}
\medskip

During the writing of this document (and the presentation afterwards, the following technologies and tools were used:
\setlist{nolistsep} \begin{itemize}[noitemsep]
	\item Ubuntu;
	\item \LaTeX, \gls{Overleaf} and Latex \gls{Beamer};
	\item \texttt{\gls{vim}} and \texttt{\gls{gedit}};
	\item \texttt{\gls{git}}, \texttt{\gls{konsole}} and \gls{GitHub};
	\item \gls{Inkscape}, \gls{Gimp} and \gls{Spectacle};
	\item \gls{Evince} Document Viewer and \gls{Calibre};
	\item \gls{Firefox} and \gls{Chromium} browsers;
\end{itemize}

\subsection{Some statistics on this document} \label{subsection:Preface/Aboutthisthesis/Somestatisticsonthisdocument}
This thesis is \pageref*{TotPages} pages long.
In absolute terms the document is \ref*{TotPages} pages long, including frontmatter's roman numbered pages.

\newcount\thetotalbibentriessumpagetotal
\newcount\thetotalbibentriesnononline% number of non webpage references
\newcount\thetotalbibentriesonline% number of online webpage references
\newcount\bibentriesaveragepagetotal% average length of a non webpage bibliography reference
\bibentriesaveragepagetotal=81% manual average from pagetotal and pages difference in bib file. For 106 pagetotals: average = 85. For 107 pages diffs without pagetotals: average = 19. Total average = 81.
\thetotalbibentriesnononline=116
\thetotalbibentriessumpagetotal=\numexpr \thetotalbibentriesnononline*\bibentriesaveragepagetotal\relax% calculate sum of totalpages as number of references * average length of a reference
\thetotalbibentriesonline=\numexpr\thetotalbibentries-\thetotalbibentriesnononline\relax

This document contains:
\setlist{nolistsep} \begin{itemize}[noitemsep]
	\item \totalfigures\ figures;
	\item \totaltables\ tables;
	\item \totallstlistings\ code listings;
	\item \totaldefinitions\ definitions,
		  \totalremarks\ remarks,
		  \totaltheorems\ theorems,
		  \totalproofs\ proofs,
		  \totalcorollarys\ corollary,
		  \totalformulas\ formulas and
		  \totalequations\ mathematical expressions;
		  \totaltodos\ todo notes.
	\item 78 acronyms,
		  34 nomenclature entries and
		  58 glossary terms;
	\item \totalparts\ parts,
		  %\totalchapters\ chapters,
		  6 chapters,
		  \totalsections\ sections,
		  \totalsubsections\ subsections and
		  \totalsubsubsections\ subsubsections;
	\item \thetotalbibentries\ bibliography references. Of these:
		\setlist{nolistsep} \begin{itemize}[noitemsep]
			\item \the\thetotalbibentriesnononline\ are books, in conference proceedings, scientific articles, tech reports, thesis documents and similar.
			Total sum of the number of pages of these references is \the\thetotalbibentriessumpagetotal\ pages.
			\item \the\thetotalbibentriesonline\ are online documentation webpages, blog posts, wiki pages and similar.
			The total sum of pages varies according to the webpage-to-PDF exportation options.
		\end{itemize}
\end{itemize}

\section{About the author} \label{section:Preface/Abouttheauthor}
\subsection[whoami]{\texttt{whoami}} \label{subsection:Preface/Abouttheauthor/WhoamI}
\myauthor, during the work for this thesis (\myyearofpublishing) is (was) a graduating student at \myinstitution\ in a \mycourse.

The supervisor professor during the work process is (was) \mysupervisor. %The examiner professor is (was) \myexaminer.

\subsection{Contacts} \label{subsection:Preface/Abouttheauthor/Getintouch}
To get in touch with the author, write to \hyperlink{mailto:\myauthoremail}{\texttt{\myauthoremail}} or connect on \gls{LinkedIn} at \url{\myauthorlinkedin}.

\newpage
\thispagestyle{empty}
