\newtheoremstyle{mytheoremstyleformat}{% name
	3pt% space above
}{%
	3pt% space below
}{%
	\itshape% font inside the body
}{%
	%\parindent% indention
}{%
	\bfseries% font in the head
}{%
	:% Punctuation after theorem head
}{%
	.5em% Space after theorem head
}{%
	% Head spec (can be empty, meaning 'normal')
	\protect\hyperref[Listofdefinitionstheoremsproofsandremarks]{\footnotesize\faListUl}%
	\ \ %
	\thmname{#1}%
	\thmnumber{ #2}%
	%\textnormal{%
	\textbf{%
		\thmnote{ (#3)}%
	}%
}%

\theoremstyle{mytheoremstyleformat}

\newtheorem{theorem}{Theorem}[chapter]% add [chapter] or [section] to reset the numeration of chapters or sections
%\crefname{theorem}{theorem}{theorems}
%\Crefname{Theorem}{Theorem}{Theorems}

\newtheorem{formula}{Formula}[chapter]% add [chapter] or [section] to reset the numeration of chapters or sections
%\crefname{formula}{formula}{formulas}
%\Crefname{Formula}{Formula}{Formulas}

\newtheorem{corollary}{Corollary}[chapter] % add [chapter], [section] or [theorem] to reset the numeration of chapters, sections or theorems. Usually a corollary is related to a theorem.
%\crefname{corollary}{corollary}{corollaries}
%\Crefname{Corollary}{Corollary}{Corollaries}

\newtheorem{lemma}[theorem]{Lemma}
%\crefname{lem}{lemma}{lemmas}
%\Crefname{Lem}{Lemma}{Lemmas}

\newtheorem{definition}{Definition}[chapter] % add [chapter] or [section] to reset the numeration of chapters or sections
%\Crefname{definition}{definition}{definitions}
%\Crefname{Definition}{Definition}{Definitions}

\newtheorem{remark}{Remark}[chapter]
%\Crefname{remark}{remark}{remarks}
%\Crefname{Remark}{Remark}{Remarks}

\let\proof\relax% solves error of amsthm defining again proof
\newtheorem{proof}{Proof}[chapter]
\let\endproof\relax% solves error of amsthm defining again proof
%\Crefname{proof}{proof}{proofs}
%\Crefname{Proof}{Proof}{Proofs}

\renewcommand{\thmcontinues}[1]{% redefinition of continued theorems, definitions, remarks etc without
	\nameref*{#1}%
	,\ %
	\hyperref[#1]{continued}%
}